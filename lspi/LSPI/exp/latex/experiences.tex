\documentclass[a3paper,oneside,12pt,landscape]{article}
%\input epsf %% at top of file

%\usepackage[dvips]{graphicx}
%\usepackage{url}
%\usepackage{tabularx}
%\usepackage{amsmath, amssymb}
%\usepackage{theorem}
%\usepackage{epsfig}
%\usepackage{color}
%\usepackage{algorithm}
%\usepackage{algorithmic}
%\usepackage{pstricks}
\usepackage[dvips,landscape]{geometry}
\usepackage[latin1]{inputenc}
\usepackage[frenchb]{babel}
\usepackage{lscape}
\usepackage{fullpage}

\setlength{\oddsidemargin}{-2cm}
\setlength{\evensidemargin}{-2cm}
\setlength{\textwidth}{40cm}
%\setlength{\topmargin}{-1cm}
%\setlength{\textheight}{28cm}


\def\R{\mbox{I\hspace{-.15em}R}}

\title{Least-Squares $\lambda$ Policy Iteration : Exp�riences}

\author{Christophe Thiery \andauthor Bruno Scherrer}

\begin{document}

\titlepage




\section{Fonctions de base polyn�miales}

\subsection{$\gamma = 0.9$ : 50 �chantillons}
\noindent
\begin{tabular}{c c}
  Point fixe & R�sidu de Bellman \\
 \input{../pol/50ech/fp_vk.tex} & \input{../pol/50ech/br_vk.tex}  \\
 \input{../pol/50ech/fp_vpi.tex} & \input{../pol/50ech/br_vpi.tex} 
\end{tabular}

\subsection{$\gamma = 0.95$ : 50 �chantillons}
\noindent
\begin{tabular}{c c}
  Point fixe & R�sidu de Bellman \\
 \input{../pol/50ech/fp_vk_g0.95.tex} & \input{../pol/50ech/br_vk_g0.95.tex}  \\
 \input{../pol/50ech/fp_vpi_g0.95.tex} & \input{../pol/50ech/br_vpi_g0.95.tex} 
\end{tabular}

\subsection{$\gamma = 0.9$ : 100 �chantillons}
\noindent
\begin{tabular}{c c}
  Point fixe & R�sidu de Bellman \\
 \input{../pol/100ech/100it_fp_vk.tex} & \input{../pol/100ech/100it_br_vk.tex}  \\
 \input{../pol/100ech/100it_fp_vpi.tex} & \input{../pol/100ech/100it_br_vpi.tex} 
\end{tabular}

%\subsection{$\gamma = 0.95$ : 100 �chantillons}
%\noindent
%\begin{tabular}{c c}
%  Point fixe & R�sidu de Bellman \\
% \input{../pol/100ech/fp_vk_g0.95.tex} & \input{../pol/100ech/br_vk_g0.95.tex}  \\
% \input{../pol/100ech/fp_vpi_g0.95.tex} & \input{../pol/100ech/br_vpi_g0.95.tex} 
%\end{tabular}

\subsection{$\gamma = 0.9$ : 200 �chantillons}
\noindent
\begin{tabular}{c c}
  Point fixe & R�sidu de Bellman \\
 \input{../pol/200ech/100it_fp_vk.tex} & \input{../pol/200ech/100it_br_vk.tex}  \\
 \input{../pol/200ech/100it_fp_vpi.tex} & \input{../pol/200ech/100it_br_vpi.tex} 
\end{tabular}

\subsection{$\gamma = 0.95$ : 200 �chantillons}
\noindent
\begin{tabular}{c c}
  Point fixe & R�sidu de Bellman \\
 \input{../pol/200ech/fp_vk_g0.95.tex} & \input{../pol/200ech/br_vk_g0.95.tex}  \\
 \input{../pol/200ech/fp_vpi_g0.95.tex} & \input{../pol/200ech/br_vpi_g0.95.tex} 
\end{tabular}

\subsection{$\gamma = 0.99$ : 200 �chantillons}
\noindent
\begin{tabular}{c c}
  Point fixe & R�sidu de Bellman \\
 \input{../pol/200ech/fp_vk_g0.99.tex} & \input{../pol/200ech/br_vk_g0.99.tex}  \\
 \input{../pol/200ech/fp_vpi_g0.99.tex} & \input{../pol/200ech/br_vpi_g0.99.tex} 
\end{tabular}

\subsection{$\gamma = 0.9$ : 500 �chantillons}
\noindent
\begin{tabular}{c c}
  Point fixe & R�sidu de Bellman \\
 \input{../pol/500ech/fp_vk.tex} & \input{../pol/500ech/br_vk.tex}  \\
 \input{../pol/500ech/fp_vpi.tex} & \input{../pol/500ech/br_vpi.tex} 
\end{tabular}

\subsection{$\gamma = 0.99$ : 500 �chantillons}
\noindent
\begin{tabular}{c c}
  Point fixe & R�sidu de Bellman \\
 \input{../pol/500ech/fp_vk_g0.99.tex} & \\% \input{../pol/500ech/br_vk_g0.99.tex}  \\
 \input{../pol/500ech/fp_vpi_g0.99.tex} & %\input{../pol/500ech/br_vpi_g0.99.tex} 
\end{tabular}

\section{Fonctions de base gaussiennes}

\subsection{$\gamma = 0.9$ : 50 �chantillons}
\noindent
\begin{tabular}{c c}
  Point fixe & R�sidu de Bellman \\
 \input{../rbf/50ech/fp_vk.tex} & \input{../rbf/50ech/br_vk.tex}  \\
 \input{../rbf/50ech/fp_vpi.tex} & \input{../rbf/50ech/br_vpi.tex} 
\end{tabular}

\subsection{$\gamma = 0.95$ : 50 �chantillons}
\noindent
\begin{tabular}{c c}
  Point fixe & R�sidu de Bellman \\
 \input{../rbf/50ech/fp_vk_g0.95.tex} & \input{../rbf/50ech/br_vk_g0.95.tex}  \\
 \input{../rbf/50ech/fp_vpi_g0.95.tex} & \input{../rbf/50ech/br_vpi_g0.95.tex} 
\end{tabular}

\subsection{$\gamma = 0.9$ : 100 �chantillons}
\noindent
\begin{tabular}{c c}
  Point fixe & R�sidu de Bellman \\
 \input{../rbf/100ech/100it_fp_vk.tex} & \input{../rbf/100ech/100it_br_vk.tex}  \\
 \input{../rbf/100ech/100it_fp_vpi.tex} & \input{../rbf/100ech/100it_br_vpi.tex} 
\end{tabular}

\subsection{$\gamma = 0.95$ : 100 �chantillons}
\noindent
\begin{tabular}{c c}
  Point fixe & R�sidu de Bellman \\
 \input{../rbf/100ech/fp_vk_g0.95.tex} & \input{../rbf/100ech/br_vk_g0.95.tex}  \\
 \input{../rbf/100ech/fp_vpi_g0.95.tex} & \input{../rbf/100ech/br_vpi_g0.95.tex} 
\end{tabular}

\subsection{$\gamma = 0.9$ : 200 �chantillons}
\noindent
\begin{tabular}{c c}
  Point fixe & R�sidu de Bellman \\
 \input{../rbf/200ech/fp_vk_g0.9.tex} & \input{../rbf/200ech/br_vk_g0.9.tex}  \\
 \input{../rbf/200ech/fp_vpi_g0.9.tex} & \input{../rbf/200ech/br_vpi_g0.9.tex} 
\end{tabular}

\subsection{$\gamma = 0.95$ : 200 �chantillons}
\noindent
\begin{tabular}{c c}
  Point fixe & R�sidu de Bellman \\
 \input{../rbf/200ech/fp_vk_g0.95.tex} & \input{../rbf/200ech/br_vk_g0.95.tex}  \\
 \input{../rbf/200ech/fp_vpi_g0.95.tex} & \input{../rbf/200ech/br_vpi_g0.95.tex} 
\end{tabular}

\subsection{$\gamma = 0.99$ : 200 �chantillons}
\noindent
\begin{tabular}{c c}
  Point fixe & R�sidu de Bellman \\
 \input{../rbf/200ech/fp_vk_g0.99.tex} & \input{../rbf/200ech/br_vk_g0.99.tex}  \\
 \input{../rbf/200ech/fp_vpi_g0.99.tex} & \input{../rbf/200ech/br_vpi_g0.99.tex} 
\end{tabular}

\subsection{$\gamma = 0.9$ : 500 �chantillons}
\noindent
\begin{tabular}{c c}
  Point fixe & R�sidu de Bellman \\
 \input{../rbf/500ech/fp_vk_g0.9.tex} & \input{../rbf/500ech/br_vk_g0.9.tex}  \\
 \input{../rbf/500ech/fp_vpi_g0.9.tex} & \input{../rbf/500ech/br_vpi_g0.9.tex} 
\end{tabular}

\subsection{$\gamma = 0.99$ : 500 �chantillons}
\noindent
\begin{tabular}{c c}
  Point fixe & R�sidu de Bellman \\
 \input{../rbf/500ech/fp_vk_g0.99.tex} & \input{../rbf/500ech/br_vk_g0.99.tex}  \\
 \input{../rbf/500ech/fp_vpi_g0.99.tex} & \input{../rbf/500ech/br_vpi_g0.99.tex} 
\end{tabular}






\end{document}


